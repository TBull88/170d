\documentclass{article}
\usepackage[utf8]{inputenc}
\newcommand*{\escape}[1]{\texttt{\textbackslash#1}}
\begin{document}
\begin{titlepage}
    \begin{center}
        \vspace*{6cm}

        
        \textbf{\Huge{Writeup}}
            
        \vspace{0.5cm}
        \textit{\large{Minorly - OS, Mod Q Practical 1B}}
        \vspace{0.5cm}
        
        \textsc{\large{170D Class: 23-001}}
            
        \vspace{7.5cm}
            
        \textsc{\small{Timothy G. Bullington}}
        \vspace{0.5cm}
        
        \textsc{\small{tmiothy.g.bullington@gmail.com}}

        \vspace{0.5cm}
        \textsc{\small{Nov 28, 2023}}

            

    \end{center}
\end{titlepage}

\section{Project Summary}

The Minorly program is designed to replicate a some of the functionality present inside the bash shell. The Minorly shell is required to handle all binary programs present in bash, along with several built-ins such as cd, IO redirect, setting environment variables, and working with background processes.\\

\section{Challenges}

The major challenge with this project was the sheer amount of implementation requirements. I feel that if we recieved this project on a Monday or Tuesday I would not have been able to complete it. In my opinion the requirements could be tailored down to just the implemetation of "cd", IO redirect, and how to implement the binaries associated with bash. The jobs requirements could be moved to extra credit and this porject would still hit all major TLO's for this block of instruction. \\

\noindent Another challenge I encountered was from the instruction itself. After taking the test and reading through this project it was pretty clear to me that I was not adequately prepped. This is not a critique on the instructor as much as it is what the material focuses on. I feel like there could be more emphasis on how signals work exactly and better code examples on how handling them is implemented. I did learn a lot while working through this project but I was only able to take the time to do extra research because we had the extra time over the holiday weekend. I fully understand that alot of this course is "exploratory learining" but a point in the right direction on what we need to be exploring would be beneficial. \\

\section{Successes}

Having said all that I do feel I have learned quite a bit about how signals and processes work with the OS. Due to the nature of this project I have had to really get into the man pages to try and understand more about tools I am using. I also found success working through the implementation of the project requirements. Finally, it was fairly easy for me to fall back into writing code in C. I was worried I would forget a lot of the syntax, but it was a pretty easy transtion back to it.\\

\pagebreak

\section{Lessons Learned}

The key take away from this project for me was that this topic is going to be critical moving forward and I need to keep learning about the OS and how to operate within this environment. OS and networking are going to be crucial for the BSLE and I need to make sure I am prepared.\\

\end{document}
